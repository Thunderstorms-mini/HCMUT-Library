\chapter{BÀI 4: BẬC PHẢN ỨNG}
    \section{Kết quả thí nghiệm}
    \subsection{Thí nghiệm 1: Tìm bậc theo $Na_{2}S_{2}O{3}$}
        \begin{center}
% Please add the following required packages to your document preamble:


    \begin{tabular}{|c|c|c|c|c|c|} 
    \hline
    \multirow{2}{*}{\textbf{TN}} & \multicolumn{2}{c|}{\textbf{Nồng độ ban đầu (M)}} & \multirow{2}{*}{$\Delta t_{1}$} & \multirow{2}{*}{$\Delta t_{2}$} & \multirow{2}{*}{$\Delta t_{TB}$}  \\ 
    \cline{2-3}
                                 & $Na_{2}S_{2}O_{3}$ & $H_{2}SO_{4}$                &                    &                    &                     \\ 
    \hline
                 1               &       4            &              8               &          108       &         78         &          93         \\ 
    \hline
                 2               &       8            &              8               &           53       &         37         &          45         \\ 
    \hline
                 3               &       28           &              8               &           28       &         28         &          28         \\
    \hline
    \end{tabular}


        \end{center}
- Từ $\Delta t_{TB}$ của TN1 và TN2 xác định m (tính mẫu): $m_{1} = \frac{lg(\frac{\Delta 1}{\Delta 2})}{log(2)} = \frac{lg(\frac{93}{45})}{lg(2)} = 1,047$\\
- Từ $\Delta t_{TB}$ của TN2 và TN3 xác định $m_{2}$: $m_{2} = \frac{lg(\frac{\Delta 2}{\Delta 3})}{log(2)} = \frac{lg(\frac{45}{28})}{log(2)} = 0,68$\\
- Bậc phản ứng theo $Na_{2}S_{2}O_{3} = \frac{m_{1} + m_{2}}{2} = \frac{1,047 + 0,68}{2} = 1,03$\\

\subsection{Thí nghiệm 2: Bậc phản ứng theo $H_{2}SO_{4}$}
\begin{center}
    \begin{tabular}{|c|c|c|c|c|c|}
         \hline \textbf{TN}&$[Na_{2}S_{2}O_{3}]$&$[H_{2}SO_{4}]$&$\Delta t_{1}$&$\Delta t_{2}$&$\Delta t_{TB}$\\
         \hline \textbf{1}&4&8&47&49&48\\
         \hline \textbf{2}&8&8&43&30&36,5\\
         \hline \textbf{3}&16&8&26&42&34\\
         \hline
    \end{tabular}
\end{center}
- Từ $\Delta t_{TB}$ của TN1 và TN2 xác định $n_{1}$: $n_{1} = \frac{lg(\frac{\Delta 1}{\Delta 2})}{log(2)} = \frac{lg(\frac{48}{36,5})}{lg(2)} = 0,40$\\
- Từ $\Delta t_{TB}$ của TN2 và TN3 xác đinh $n_{2}$: $n_{2} = \frac{lg(\frac{\Delta 2}{\Delta 3})}{log(2)} = \frac{lg(\frac{36,5}{34})}{log(2)} = 0,10$\\\
- Bậc phản ứng theo $H_{2}SO_{4} = \frac{n_{1}+n_{2}}{2} = \frac{0,4 + 0,1}{2} = 0,25$\\

\newpage 

\section{Trả lời câu hỏi}

\textbf{1.} Trong TN trên, nồng độ của $Na_{2}S_{2}O_{3}$ và của $H_{2}SO_{4}$ đã ảnh hưởng thế nào lên vận tốc phản ứng? Viết lại biểu thức tính vận tốc phản ứng. Xác định bậc của phản ứng.\\
- Nồng độ của $Na_{2}S_{2}O_{3}$ tỉ lệ thuận với tốc độ phản ứng.\\
- Nồng độ của $H_{2}SO_{4}$ hầu như không ảnh hưởng đến tốc độ phản ứng.\\
- Biểu thức tính tốc độ phản ứng: $v = k[Na_{2}S_{2}O_{3}]^{m}[H_{2}SO_{4}]^{n}$, trong đó m,n là hằng số dương xác định bằng thực nghiệm.\\
- Bậc phản ứng: $m+n$.\\

\textbf{2.} Cơ chế phản ứng trên có thể được viết như sau:
\begin{center}
    $H_{2}SO_{4} + Na_{2}S_{2}O_{3} \to Na_{2}SO_{4} + H_{2}S_{2}O_{3}$ (1)\\
    $H_{2}S_{2}O_{3} \to H_{2}SO_{3} + S \downarrow$ \hspace{2.9cm}(2)\\
\end{center}
Dựa vào kết quả TN có thể kết luận phản ứng (1) hay (2) là phản ứng quyết định vận tốc phản ứng tức là phản ứng xảy ra chậm nhất không? Tại sao? Lưu ý trong các TN trên, lượng axit $H_{2}SO_{4}$ luôn luôn dư so với $Na_{2}S_{2}O_{3}$.\\
- Phản ứng (1) là phản ứng trao đổi ion nên tốc độ phản ứng xảy ra nhanh.\\
- Phản ứng (2) xảy ra chậm hơn.\\
$\to$ Phản ứng (2) quyết định tốc độ phản ứng và là phản ứng xảy ra chậm nhất vì bậc của phản ứng là bậc của phản ứng (2).\\

\textbf{3.} Dựa trên cơ sở của phương pháp TN thì vận tốc xác định được trong các TN trên được xem là vận tốc trung bình hay vận tốc tức thời?\\
- Dựa trên cơ sở của phương pháp thí nghiệm thì vận tốc xác định được trong các thí nghiệm được xem là vận tốc tức thời vì vận tốc phản ứng được xác định bằng tỉ số $\Delta C/\Delta t$. Vì $\Delta C \approx 0$ (do lưu huỳnh thay đổi không đáng kể nên $\Delta C \approx dC$).\\

\textbf{4.} Thay đổi thứ tự cho $H_{2}SO_{4}$ và $Na_{2}S_{2}O_{3}$ thì bậc phản ứng có thay đổi không? Tại sao?\\
- Bậc phản ứng không thay đổi vì bậc phản ứng chỉ phụ thuộc vào nhiệt độ và bản chất của phản ứng mà không phụ thuộc vào quá trình tiến hành.