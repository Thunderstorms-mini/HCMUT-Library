\chapter{Bài 8: PHÂN TÍCH THỂ TÍCH}

\section{Kết quả thí nghiệm}

\subsection{Thí nghiệm 1: Xác định đường cong chuẩn độ HCl bằng NaOH}
\begin{center}
    \begin{figure}[htp]
        \begin{center}
            \includegraphics[scale=1]{Dothi.png}
        \end{center}
    \end{figure}
\end{center}
Xác định:
\begin{itemize}
    \item pH điểm tương đương là 7.
    \item Bước nhảy pH: từ 3,36 đến pH 10,56.
\end{itemize}

\newpage

\subsection{Thí nghiệm 2: Tìm nồng độ HCl với phenolphtalein}
\begin{center}
\begin{tabular}{|c|c|c|c|c|c|}
    \hline \textbf{Lần} & $V_{HCl}$ (ml) & $V_{NaOH}$ (ml) & $C_{NaOH}$ (N) & $C_{HCl}$ (N) & Sai số\\
    \hline       1      &   10           &      10,3       &      0,1       &    0,103      & 0,003 \\
    \hline       2      &   10           &                 &      0,1       &               &       \\
    \hline
\end{tabular}
\end{center}

\subsection{Thí nghiệm 3: Tìm nồng độ HCl với Metyl da cam}
\begin{center}
    \begin{tabular}{|c|c|c|c|c|c|}
        \hline \textbf{Lần} & $V_{HCl}$ (ml) & $V_{NaOH}$ (ml) & $C_{NaOH}$ (N) & $C_{HCl}$ (N) & Sai số\\
        \hline       1      &   10           &         8       &      0,1       &        0,08   &  0,02 \\
        \hline       2      &   10           &                 &      0,1       &               &       \\
        \hline
    \end{tabular}
\end{center}

\subsection{Thí nghiệm 4: Tìm nồng độ $CH_{3}COOH$}
\begin{center}
    \begin{tabular}{|c|c|c|c|c|c|}
        \hline \textbf{Lần} & \textbf{Chất chỉ thị} & $V_{CH_3COOH}$ (ml) & $V_{NaOH}$ (ml) & $C_{NaOH}$ (N) & $C_{CH_3COOH}$ (N)\\
        \hline      1       &  Phenol phtalein      &          10         &       8,5       &     0,1        &      0,085        \\
        \hline      2       &  Metyl orange         &          10         &       0,5       &     0,1        &      0,005        \\
        \hline
    \end{tabular}
\end{center}


\section{Trả lời câu hỏi}

\textbf{1.} Khi thay đổi nồng độ HCl và NaOH, đường cong chuẩn độ có thay đổi không, tại sao? \\
- Khi thay đổi nồng độ HCl và NaOH thì đường cong chuẩn độ vẫn không thay đổi vì đượng lượng phản ứng của các chất không thay đổi, chỉ có bước nhảy là thay đổi. Nếu dùng nồng độ nhỏ thì bước nhảy nhỏ và ngược lại.\\

\textbf{2.} Việc xác định nồng độ axit HCl trong các thí nghiệm 2 và 3 cho kết quả chính xác hơn, tại sao?\\
- Việc xác định nồng độ axit HCl trong các thí nghiệm 2 và 3 thì thí nghiệm 2 cho ta kết quả chính xác hơn. Vì phenol phtalein có bước nhảy pH trong khoảng 8-10 còn metyl organe là 3.1-4.4 mà điểm tương đương của hệ là 7.\\

\textbf{3.} Từ kết quả thí nghiệm 4, việc xác định nồng độ dung dịch axit acetic bằng chỉ thị màu nào chính xác hơn, tại sao?\\
- Từ kết quả thí nghiệm 4, việc xác định nồng độ dung dịch axit acetic bằng phenol phtalein  cho ta kết quả chính xác hơn. Vì phenol phtalein có bước nhảy pH trong khoảng 8-10 còn metyl organe là 3.1-4.4 mà điểm tương đương của hệ lớn hơn 7.\\

\textbf{4.} Trong phép phân tích thể tích, nếu đổi vị trí của NaOH và axit thì kết quả có thay đổi không, tại sao?\\
- Trong phép phân tích thể tích nếu thay đổi vị trí của NaOH và axit thì kết quả vẫn không thay đổi vì bản chất của phản ứng vẫn là phản ứng trung hòa và chất chỉ thị cũng vẫn sẽ đổi màu tại điểm tương đương.

\begin{center}
    \textbf{(HẾT PHẦN BÁO CÁO)}
\end{center}
\newcommand{\xfill}[2][1ex]{{%
  \dimen0=#2\advance\dimen0 by #1
  \leaders\hrule height \dimen0 depth -#1\hfill%
}}

\ \xfill{1pt} \
\newpage

* PHÂN CÔNG BÀI LÀM *\\

\begin{tabular}{|c|c|}
    \hline Tên thành viên & Tên bài làm \\
    \hline Lâm Hồng Phúc - 2011843& Bài 2, bài 4\\
    \hline Nguyễn Hoàng Phúc - 2014172& Bài 4, bài 8\\
    \hline
\end{tabular}